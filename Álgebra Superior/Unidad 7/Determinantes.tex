\documentclass[11pt]{beamer}
\usetheme{Madrid}
\usepackage[utf8]{inputenc}
\usepackage[spanish]{babel}
\usepackage{amsmath}
\usepackage{amsfonts}
\usepackage{amssymb}
\usepackage{graphicx}
\usepackage{amsthm}
\newtheorem{defi}{Definición}
\newtheorem{eje}{Ejercicio}
\newtheorem{ejem}{Ejemplo}
\newtheorem{axiom}{Axioma}
\newtheorem{teor}{Teorema}
\usepackage{lipsum}
\author{Adriana Dávila Santos}
\title{Determinantes}
%\setbeamercovered{transparent} 
%\setbeamertemplate{navigation symbols}{} 
%\logo{} 
%\institute{} 
%\date{} 
%\subject{} 
\begin{document}

\begin{frame}
\titlepage
\end{frame}

%\begin{frame}
%\tableofcontents
%\end{frame}

\begin{frame}{• Objetivo particular: }
El alumno establecerá el determinante de una matriz, identificará las propiedades de los
determinantes, los calculará aplicando sus propiedades y los utilizará para calcular inversas de
matrices y resolver sistemas de ecuaciones.
\end{frame}

\begin{frame}
\frametitle{Concepto de determinante}
\framesubtitle{Cálculo de determinantes}
\begin{defi}
A cada matriz cuadrada $A$ se le asocia un escalar particular denominado \textbf{determinante} de $A$, denotado por $|A|$ o por $det(A)$. Este escalar permite caracterizar algunas propiedades de la matriz.
\end{defi}
\end{frame}

\begin{frame}
\frametitle{Concepto de determinante}
\framesubtitle{Cálculo de determinantes}
Cálculo de determinantes:\\
\begin{itemize}
\item El determinante de una matriz A de orden 1 es el elemento $a_{11}$, es decir, $det(A)=a_{11}$
\item El determinante de una matriz A de orden 2 es: $det(A) = a_{11}\cdot a_{22} - a_{12}\cdot a_{21}$
\item El determinante de una matriz A de orden 3 es: 
$det(A) = a_{33}a_{22}a_{11}+a_{32}a_{21}a_{13}+a_{23}a_{12}a_{31}-a_{31}a_{22}a_{13}-a_{32}a_{23}a_{11}-a_{21}a_{12}a_{33}$
\end{itemize}
\end{frame}

\begin{frame}
\frametitle{Concepto de determinante}
\framesubtitle{Cálculo de determinantes}
\begin{ejem}
Calcula el determinante de las siguientes matrices:\\ \hspace{0cm} \\
\begin{itemize}
\item ${\displaystyle A = {\begin{bmatrix}10\end{bmatrix}}}\rightarrow$ $det(A)=10$
\item ${\displaystyle B = {\begin{bmatrix}1&2\\2&3\end{bmatrix}}}\rightarrow$ $det(B)=(3)(1)-(2)(2)=-1$
\item ${\displaystyle C = {\begin{bmatrix}1&0&2\\2&-1&3\\4&1&8\end{bmatrix}}}$\\ 
$\rightarrow det(C) = (8)(-1)(1)+(3)(0)(4)+(1)(2)(2)-(4)(-1)(2)-(1)(3)(1)-(2)(0)(8) = 1$
\end{itemize}
\end{ejem}
\end{frame}

\begin{frame}
\frametitle{Menores y cofactores}
\framesubtitle{Definiciones}
\begin{defi}
Se le llama \textbf{menor} al determinante de una submatriz de una matriz A obtenida al eliminar una o más filas o columnas. Lo denotaremos como $M_{ij}$
\end{defi}
En nuestro caso obtendremos el menor $(i,j)$ al eliminar la fila y columna donde se encuentre el elemento $a_{ij}$
\begin{ejem}
Obten los menores indicados dada la matriz A: \\ \hspace{0cm} \\
\begin{center}
${\displaystyle A = {\begin{bmatrix}1&0&2\\2&-1&3\\4&1&8\end{bmatrix}}}$
\end{center}
\end{ejem}
\end{frame}

\begin{frame}
\frametitle{Menores y cofactores}
\framesubtitle{Definiciones}
\begin{ejem}
\begin{itemize}
\item $M_{11}$ \\
\begin{center}
${\displaystyle {\begin{bmatrix}-1&3\\1&8\end{bmatrix}}} ~ \rightarrow M_{11} = -8-3=-11$
\end{center}
\item $M_{13}$ \\
\begin{center}
${\displaystyle {\begin{bmatrix}2&-1\\4&1\end{bmatrix}}} ~ \rightarrow M_{13} = 2+4=6$
\end{center}
\item $M_{22}$ \\
\begin{center}
${\displaystyle {\begin{bmatrix}1&2\\4&8\end{bmatrix}}} ~ \rightarrow M_{22} = 8-8=0$
\end{center}
\end{itemize}
\end{ejem}
\end{frame}

\begin{frame}
\frametitle{Menores y cofactores}
\framesubtitle{Definiciones}
\begin{defi}
Sea una matriz cuadrada A, el \textbf{cofactor} del elemento $a_{ij}$ denotado como $c_{ij}$ se obtiene de la siguiente forma: \\
\begin{center}
$c_{ij} = (-1)^{i+j}M_{ij}$
\end{center}
\end{defi}
\begin{ejem}
Continuando con el ejemplo anterior de menores, obtenga los siguientes cofactores:\\
\begin{itemize}
\item $c_{11} = (-1)^{1+1}(-11) = -11$
\item $c_{13} = (-1)^{1+3}(6) = 6$
\item $c_{22} = (-1)^{2+2}(0) = 0$
\end{itemize}
\end{ejem}
\end{frame}

\begin{frame}
\frametitle{Menores y cofactores}
\framesubtitle{Determinantes}
Ya vimos como obtener el determinante de una matriz de orden 1,2 y 3.
El siguiente método sirve para obtener determinantes de orden $n$, $\forall n\in \mathbb{N}$.\\
Para obtener el valor de un determinante utilizando cofactores, se procede como se indica a  continuación.
\begin{itemize}
\item Se escoge cualquier fila o cualquier columna
\item Se calcula el cofactor $c_{ij}$ de cada elemento de la fila o columna escogida
\item Se multiplican cada elemento de la fila o columna por su respectivo cofactor de la siguiente forma: $a_{ij}c_{ij}$.
\item Se suman los productos obtenidos en el punto anterior y el resultado obtenido es el valor del determinante.
\end{itemize}
\end{frame}

\begin{frame}
\frametitle{Menores y cofactores}
\framesubtitle{Determinantes}
\begin{ejem}
Calcule el determinante de la siguiente matriz:\\ \hspace{0cm} \\
\begin{center}
${\displaystyle A = {\begin{bmatrix}0&4&-2&4\\-6&2&10&0\\5&8&-5&2\\0&-2&1&0\end{bmatrix}}}$
\end{center}
Una consideración importante es usar el renglón o columna que contenga más ceros en sus elementos, pues al ser una multiplicación, podemos evitar el cálculo de cofactores que al final se van a anular.\\
Así que por conveniencia podemos escoger la tercer columna de o el tercer renglón indistintamente, en nuestro caso usaremos la tercer fila.
\end{ejem}
\end{frame}

\begin{frame}
\frametitle{Menores y cofactores}
\framesubtitle{Determinantes}
\begin{ejem}
Ya que decidimos la fila a trabajar, definimos la siguiente fórmula:\\
\begin{center}
$det(A)=(-1)^{4+1}(a_{41})(c_{41})+(-1)^{4+2}(a_{42})(c_{42})+(-1)^{4+3}(a_{43})(c_{43})+(-1)^{4+4}(a_{44})(c_{44})$\\ \hspace{0cm} \\
$det(A)=(-1)^{5}(0)(c_{41})+(-1)^{6}(-2)(c_{42})+(-1)^{7}(1)(c_{43})+(-1)^{8}(0)(c_{44})$\\ \hspace{0cm} \\
$det(A)=(-2)(c_{42})-(1)(c_{43}) = -2c_{42}-c_{43}$ \\ \hspace{0cm} \\ 
\end{center}
Ahora que ya simplificamos la expresión vemos que sólo debemos encontrar dos cofactores, que en este caso son dos determinantes de una matriz de orden 3.
\end{ejem}
\end{frame}

\begin{frame}
\frametitle{Menores y cofactores}
\framesubtitle{Determinantes}
\begin{ejem}
$M_{42} = \left|
\begin{array}{ccc}
0 & -2 & 4\\
-6 & 10 & 0\\
5 & -5 & 2\\
\end{array}
\right| = -104 ~ \rightarrow c_{42}=(-1)^{4+2}(-104) = -104$ \\ \hspace{0cm} \\

$M_{43} = \left|
\begin{array}{ccc}
0 & 4 & 4\\
-6 & 2 & 0\\
5 & 8 & 2\\
\end{array}
\right| = -184 ~ \rightarrow c_{43}=(-1)^{4+3}(-184) = 184$ \\ \hspace{0cm} \\
$\therefore det(A) = -2(-104)+184 = 392$
\end{ejem}
\end{frame}

\begin{frame}
\frametitle{Matrices triangulares y diagonales}
\framesubtitle{Determinantes}
\begin{defi}
Se dice que una matriz cuadrada A es \textbf{triangular} si los elementos por debajo o por arriba de la diagonal principal son todos cero, es \textbf{triangular superior} si están por debajo y es \textbf{triangular inferior} si están por arriba.
\end{defi}
\begin{defi}
Una matriz cuadrada A es una \textbf{matriz diagonal} si es tanto triangular superior como inferior, es decir, todos los elementos fuera de la diagonal principal son cero.
\end{defi}
\end{frame}

\begin{frame}
\frametitle{Matrices triangulares y diagonales}
\framesubtitle{Determinantes}
Calcular el determinante de una matriz triangular o de una matriz diagonal es una tarea sumamente sencilla, ya que el determinante es la multiplicación de los elementos de la diagonal.
\begin{ejem}
Calcule el determinante de las siguientes matrices: \\ \hspace{0cm} \\
\begin{itemize}
\item ${\displaystyle A = {\begin{bmatrix}1&4&-2&4\\0&2&10&0\\0&0&-5&2\\0&0&0&8\end{bmatrix}}} ~ \rightarrow 
det(A) = 1 \cdot 2 \cdot (-5) \cdot 8 = -80$
\item ${\displaystyle B = {\begin{bmatrix}1&0&0&0\\7&3&0&0\\6&6&5&0\\1&4&0&9\end{bmatrix}}} ~ \rightarrow 
det(B) = 1 \cdot 3 \cdot 5 \cdot 9 = 135$
\end{itemize}
\end{ejem}
\end{frame}

\begin{frame}
\frametitle{Matrices triangulares y diagonales}
\framesubtitle{Determinantes}
\begin{ejem}
\begin{itemize}
\item ${\displaystyle C = {\begin{bmatrix}1&0&0&0\\0&12&0&0\\0&0&-5&0\\0&0&0&1\end{bmatrix}}} ~ \rightarrow 
det(C) = 1 \cdot 12 \cdot (-5) \cdot 1 = -60$
\end{itemize}
Vemos que la matriz A es un ejemplo de matriz triangular superior, B es una matriz triangular inferior y C es una matriz diagonal.
\end{ejem}
\end{frame}

\begin{frame}
\frametitle{Propiedades de los determinantes}
Propiedades: \\
\begin{itemize}
\item Una matriz cuadrada con una fila o una columna en la que todos los elementos son nulos tiene un determinante igual a cero.
\item El determinante de una matriz con dos filas o dos columnas iguales es nulo.
\item Cuando dos filas o dos columnas de una matriz son proporcionales entre sí (una se puede obtener multiplicando la otra por un factor), su determinante es cero.
\item Al intercambiar dos filas o dos columnas de una matriz, su determinante cambia de signo.
\item Al multiplicar todos los elementos de una fila o una columna de una matriz por un número, el determinante de la matriz resultante es igual al de la original multiplicado por ese mismo número.
\item Cuando a una fila (o columna) de una matriz se le suma o resta una combinación lineal de otras filas (o columnas), el valor de su determinante no se altera.
\end{itemize}
\end{frame}

\begin{frame}
\frametitle{Propiedades de los determinantes}
\framesubtitle{Cálculo de determinantes}
Con todos los conceptos y propiedades vistas hasta ahora, podemos ver que otra forma de calcular determinantes de una matriz es aplicarle operaciones elementales por renglón para llevarlo a una forma escalonada, recordemos que la forma escalonada es una matriz triangular superior, por lo que su determinante será el producto de los elementos de la diagonal principal.
\end{frame}

\begin{frame}
\frametitle{Propiedades de los determinantes}
\framesubtitle{Cálculo de determinantes}
\begin{ejem}
Calcula el determinante de la siguiente matriz usando operaciones elementales por renglón
\begin{center}
${\displaystyle A = {\begin{bmatrix}1&0&2\\2&-1&3\\4&1&8\end{bmatrix}}}$\\ \hspace{0cm} \\
\end{center}
${\displaystyle {\begin{bmatrix}1&0&2\\2&-1&3\\4&1&8\end{bmatrix}}} 
~ R_2 \rightarrow R_2 -2R_1 ~ {\displaystyle {\begin{bmatrix}1&0&2\\0&-1&-1\\4&1&8\end{bmatrix}}}$\\ \hspace{0cm} \\

${\displaystyle {\begin{bmatrix}1&0&2\\0&-1&-1\\4&1&8\end{bmatrix}}} 
~ R_3 -4R_1 \rightarrow ~ {\displaystyle {\begin{bmatrix}1&0&2\\0&-1&-1\\0&1&0\end{bmatrix}}}$\\ \hspace{0cm} \\
\end{ejem}
\end{frame}

\begin{frame}
\frametitle{Propiedades de los determinantes}
\framesubtitle{Cálculo de determinantes}
\begin{ejem}
${\displaystyle {\begin{bmatrix}1&0&2\\0&-1&-1\\0&1&0\end{bmatrix}}} 
~ R_3 +R_2 \rightarrow ~ {\displaystyle {\begin{bmatrix}1&0&2\\0&-1&-1\\0&0&-1\end{bmatrix}}}$\\ \hspace{0cm} \\
$\therefore d(A) = 1$
\end{ejem}
\end{frame}

\begin{frame}
\frametitle{Determinantes, matrices y sistemas de ecuaciones}
\framesubtitle{Matriz adjunta de una matriz cuadrada}
\begin{defi}
Dada una matriz cuadrada A, la matriz \textbf{adjunta} denotada como \textbf{adj(A)} es una matriz del mismo orden que A y se obtiene al sustituir los elementos $a_{ij}$ por su cofactor correspondientes $c_{ij}$.
\end{defi}
\begin{ejem}
${\displaystyle A = {\begin{bmatrix}1&2\\-1&3\end{bmatrix}}} \rightarrow 
{\displaystyle adj(A) = {\begin{bmatrix}3&1\\-2&1\end{bmatrix}}}$\\ \hspace{0cm} \\
\begin{itemize}
\item $M_{11} = 3 \rightarrow c_{11} = 3$
\item $M_{12} = -1 \rightarrow c_{12} = 1$
\item $M_{21} = 2 \rightarrow c_{21} = -2$
\item $M_{22} = 1 \rightarrow c_{22} = 1$
\end{itemize}
\end{ejem}
\end{frame}

\begin{frame}
\frametitle{Determinantes, matrices y sistemas de ecuaciones}
\framesubtitle{Cálculo de la matriz inversa por medio de la adjunta}
Con anterioridad ya hemos definido una matriz inversa, ahora veremos que existe un método para obtenerla usando la matriz adjunta mediante la siguiente fórmula:\\
\begin{center}
Sea la matriz A, entonces $A^{-1} = \frac{1}{det(A)}\cdot adj(A^T)$
\end{center}
Aquí es fácil notar que si det(A) = 0, entonces $A^{-1} \nexists$
\end{frame}

\begin{frame}
\frametitle{Determinantes, matrices y sistemas de ecuaciones}
\framesubtitle{Cálculo de la matriz inversa por medio de la adjunta}
\begin{ejem}
Siguiendo el ejemplo anterior dónde: \\ \hspace{0cm} \\
${\displaystyle A = {\begin{bmatrix}1&2\\-1&3\end{bmatrix}}}, 
{\displaystyle adj(A) = {\begin{bmatrix}3&1\\-2&1\end{bmatrix}}} \rightarrow 
{\displaystyle adj(A^T) = {\begin{bmatrix}3&-2\\1&1\end{bmatrix}}}$\\ \hspace{0cm} \\
$det(A) = (3)(1)-(-1)(2) = 5$\\ \hspace{0cm} \\
$\rightarrow A^{-1} = \frac{1}{det(A)}\cdot adj(A^T) 
= \frac{1}{5} {\displaystyle {\begin{bmatrix}3&-2\\1&1\end{bmatrix}}}$\\ \hspace{0cm} \\
$\therefore \displaystyle A^{-1} = {\begin{bmatrix}\frac{3}{5}&-\frac{2}{5}\\\frac{1}{5}&\frac{1}{5}\end{bmatrix}}$
\end{ejem}
\begin{eje}
Demostrar que $adj(A^T) = adj(A)^T$
\end{eje}
\end{frame}

\begin{frame}
\frametitle{Determinantes, matrices y sistemas de ecuaciones}
\framesubtitle{Sistemas de ecuaciones}
Ya hemos visto que las matrices son muy útiles para resolver sistemas de ecuaciones, ahora veremos una nueva forma de resolver estos sistemas con ayuda de la matriz inversa.
Lo explicaremos siguiendo el ejemplo anterior: \\ \hspace{0cm} \\
${\displaystyle A = {\begin{bmatrix}1&2\\-1&3\end{bmatrix}}}, 
\displaystyle A^{-1} = {\begin{bmatrix}\frac{3}{5}&-\frac{2}{5}\\\frac{1}{5}&\frac{1}{5}\end{bmatrix}}$\\ \hspace{0cm} \\
Suponiendo que A es la matriz asociada a un sistema de ecuaciones lineales cuyos términos independientes del lado derecho de la ecuación son 15 y 5 respectivamente, de dos incógnitas y dos ecuaciones, sabemos que tenemos la siguiente igualdad: $Ax=b$, donde $x$ es la matriz de incógnitas y $b$ la matriz de términos independientes. Podemos replantearla de la siguiente forma: \\ \hspace{0cm} \\
$A^{-1}Ax = A^{-1}b \rightarrow I_2x = A^{-1}b$\\
$\therefore x = A^{-1}b$
\end{frame}

\begin{frame}
\frametitle{Determinantes, matrices y sistemas de ecuaciones}
\framesubtitle{Sistemas de ecuaciones}
Así que para encontrar los valores de la matriz $x$ basta con multiplicar a la inversa por la matriz de términos independientes. 
\\ \hspace{0cm} \\
$x = \displaystyle {\begin{bmatrix}\frac{3}{5}&-\frac{2}{5}\\\frac{1}{5}&\frac{1}{5}\end{bmatrix}} 
\displaystyle {\begin{bmatrix}15\\5\end{bmatrix}} = {\begin{bmatrix}7\\4\end{bmatrix}}$\\ \hspace{0cm} \\
$\therefore x_1 = 7, ~ x_2 = 4$ \\ \hspace{0cm} \\
Ahora conocemos otra forma de resolver sistemas de ecuaciones lineales cuya matriz asociada sea cuadrada.
\end{frame}

\begin{frame}
\frametitle{Determinantes, matrices y sistemas de ecuaciones}
\framesubtitle{Regla de Cramer}
Anteriormente vimos cómo calcular determinantes de matrices de diferentes modos, la regla o método de Cramer usa ese concepto para resolver sistemas de ecuaciones lineales cuya matriz de coeficientes asociada sea cuadrada. El método es bastante sencillo y nos da la siguiente igualdad: \\ \hspace{0cm} \\
\begin{itemize}
\item $x_i = \frac{D_{x_i}}{D}$
\end{itemize}
Donde $D$ es el determinante de la matriz asociada y $D_{x_i}$ es el determinante de la matriz resultante de cambiar el vector de valores de la variable $x_i$ por el vector de valores independientes $b$ 
\end{frame}

\begin{frame}
\frametitle{Determinantes, matrices y sistemas de ecuaciones}
\framesubtitle{Regla de Cramer}
\begin{ejem}
Resuelva el siguiente sistema de ecuaciones lineales por regla de Cramer: \\
\begin{center}
$2x_1 + x_2 - 2x_3 = 1$\\
$3x_1 - 2x_2 + x_3 = 0$\\
$x_1 + 3x_2 - 1x_3 = 2$\\ 
\end{center}
Sean las siguientes matrices asociadas al sistema: \\ \hspace{0cm} \\
$\displaystyle A = {\begin{bmatrix}2&1&-2\\3&-2&1\\1&3&-1\end{bmatrix}}$,
$\displaystyle b = {\begin{bmatrix}1\\0\\2\end{bmatrix}}$\\ \hspace{0cm} \\
$det(A) = D = 4+1-18-4-6+3 = -20$\\
\end{ejem}
\end{frame}

\begin{frame}
\frametitle{Determinantes, matrices y sistemas de ecuaciones}
\framesubtitle{Regla de Cramer}
\begin{ejem}
Ahora diremos que $A_i$ será la matriz resultante de cambiar los valores de la columna $i$ por los valores del vector columna $b$: \\ \hspace{0cm} \\
$\displaystyle A_1 = {\begin{bmatrix}1&1&-2\\0&-2&1\\2&3&-1\end{bmatrix}}$,
$\displaystyle A_2 = {\begin{bmatrix}2&1&-2\\3&0&1\\1&2&-1\end{bmatrix}}$,
$\displaystyle A_3 = {\begin{bmatrix}2&1&1\\3&-2&0\\1&3&2\end{bmatrix}}$\\ \hspace{0cm} \\
$det(A_i) = D_{x_i}$ 
\begin{itemize}
\item $D_{x_1} = 2+2+0-8-3-0 = -7$
\item $D_{x_2} = 0+1-12-0-4+3 = -12$
\item $D_{x_3} = -8+9+0+2-0-6 = -3$
\end{itemize}
\end{ejem}
\end{frame}

\begin{frame}
\frametitle{Determinantes, matrices y sistemas de ecuaciones}
\framesubtitle{Regla de Cramer}
\begin{ejem}
\begin{itemize}
\item $x_1 = \frac{-7}{-20} = \frac{7}{20}$
\item $x_2 = \frac{-12}{-20} = \frac{3}{5}$
\item $x_3 = \frac{-3}{-20} = \frac{3}{20}$
\end{itemize}
De este modo es cómo resolvemos sistemas de ecuaciones lineales por regla de Cramer.
\end{ejem}
\end{frame}

\end{document}