\documentclass[11pt]{beamer}
\usetheme{Madrid}
\usepackage[utf8]{inputenc}
\usepackage[spanish]{babel}
\usepackage{amsmath}
\usepackage{amsfonts}
\usepackage{amssymb}
\usepackage{graphicx}
\usepackage{amsthm}
\newtheorem{defi}{Definición}
\newtheorem{eje}{Ejercicio}
\newtheorem{ejem}{Ejemplo}
\newtheorem{axiom}{Axioma}
\usepackage{Lipsum}

\author{Adriana Dávila Santos}
\title{Números y Operaciones}
%\setbeamercovered{transparent} 
%\setbeamertemplate{navigation symbols}{} 
%\logo{} 
%\institute{} 
%\date{} 
%\subject{} 
\begin{document}

\begin{frame}
\titlepage
\end{frame}

%\begin{frame}
%\tableofcontents
%\end{frame}

\begin{frame}{• Objetivo particular}
El alumno distinguirá los diferentes tipos de números: naturales, enteros, racionales y reales e
identificará las propiedades de la suma y el producto en estos números.
\end{frame}
\begin{frame}
\frametitle{Números naturales}
\framesubtitle{Suma y producto de los números naturales como operaciones binarias}
\begin{defi}
Al conjunto discreto de números positivos que permiten contar u ordenar los elementos
de otro conjunto, se le conoce como \textbf{números naturales} y se denota de la siguiente forma:\\
\begin{center}
$\mathbb{N} = \{1,2,3,...\}$
\end{center}
\end{defi}
\begin{defi}
Sea una operación matemática que necesita como argumentos dos operandos y un operador para
devolver un valor, se le conoce como \textbf{operación binaria}.
\end{defi}
\end{frame}

\begin{frame}
\frametitle{Números naturales}
\framesubtitle{Suma y producto de los números naturales como operaciones binarias}
La operación suma o adición en los números naturales requiere de dos operandos (números):
$a,b \in \mathbb{N}$ y el operador '+' para devolver un valor $c$, de este modo
vemos que cumple con la definición de operación binaria.\hspace{2cm}

NOTA: Además podemos observar que $\forall ~ a,b \in \mathbb{N} ~ \& ~ a + b = c \rightarrow c\in \mathbb{N}$,
a esta propiedad se le llama \textbf{cerradura} y de manera general signífica que la suma de dos 
números naturales siempre devuelve como resultado un número natural.
\end{frame}

\begin{frame}
\frametitle{Números naturales}
\framesubtitle{Suma y producto de los números naturales como operaciones binarias}
La operación producto o multiplicación en los números naturales requiere de dos operandos (números):
$a,b \in \mathbb{N}$ y el operador $\cdot$ para devolver un valor $c$, de este modo
vemos que cumple con la definición de operación binaria.\\ \hspace{2cm}

NOTA: Además podemos observar que $\forall ~ a,b \in \mathbb{N} ~ \& ~ a \cdot b = c \rightarrow c\in \mathbb{N}$,
a esta propiedad se le llama \textbf{cerradura} y de manera general signífica que el producto de dos 
números naturales siempre devuelve como resultado un número natural.
\end{frame}


\begin{frame}
\frametitle{Números naturales}
\framesubtitle{Conmutatividad}
\begin{defi}
La conmutaividad es la propiedad que poseen algunas operaciones binarias y permite cambiar el orden de los operandos
sin afectar el resultado.
\end{defi}
\begin{itemize}
\item Conmutatividad en la suma\\
$a + b = b + a = c,~\forall ~ a,b \in \mathbb{N}$\\ \hspace{2cm}
\item Conmutatividad en el producto\\
$a \cdot b = b \cdot a = c,~\forall ~ a,b \in \mathbb{N}$
\end{itemize}
\end{frame}

\begin{frame}
\frametitle{Números naturales}
\framesubtitle{Asociatividad}
\begin{defi}
La asociatividad es la propiedad que poseen algunas operaciones binarias e indica que dados 3 o más elementos de un conjunto,
se pueden operar de manera indistinta en su orden, agrupando por pares de operandos.
\end{defi}
\begin{itemize}
\item Asociatividad en la suma\\
$(a + b) + c = a + (b + c) = d,~\forall ~ a,b,c \in \mathbb{N}$\\ \hspace{2cm}
\item Asociatividad en el producto\\
$(a \cdot b) \cdot c = a \cdot (b \cdot c) = d,~\forall ~ a,b,c \in \mathbb{N}$
\end{itemize}
\end{frame}

\begin{frame}
\frametitle{Números naturales}
\framesubtitle{Distributividad}
\begin{defi}
Dadas las operaciones binarias cualesquiera $\oplus,~\otimes$ y el conjunto $\mathbb{N}$, se dice que $\otimes$ es distributiva sobre $\oplus$ si $\forall ~ x,y,z \in \mathbb{N}$ se cumple que $x \otimes (y \oplus z) = (x \otimes y) \oplus (x \otimes z)~\&~
(y \oplus z) \otimes x = (y \otimes x) \oplus (z \otimes x)$
\end{defi}
En el caso de los números naturales, el producto es distributivo sobre la suma ya que se cumple la siguiente expresión:
\begin{center}
$a \cdot (b + c) = (a \cdot b) + (a \cdot c) = (b + c) \cdot a = (b \cdot a) + (c \cdot a)$
\end{center}
\end{frame}

\begin{frame}
\frametitle{Números naturales}
\begin{ejem}
Sea la operación binaria AND ($\&$) que establece una operación lógica bajo el conjunto $S = \{0,1\}$, tendremos la siguiente tabla de resultados (recordemos que solo es verdadero o 1 cuando los dos operandos son verdaderos, en este caso 1).
\begin{center}
\begin{tabular}{| r | c | l |}
\hline
\textbf{\&} & \textbf{0} & \textbf{1}	\\ \hline
\textbf{0} & $0$ & $0$\\
\textbf{1} & $0$ & $1$\\ \hline
\end{tabular}
\end{center}
Es fácil ver que existe la propiedad conmutativa dada esta operación bajo este conjunto pues $1 ~ \& ~ 0 = 0 ~ \& ~ 1 = 0$\\
También se posee la propiedad asociativa ya que no importa el orden en que agrupemos y operemos, pues sólo será 1 cuando todos los elementos que estamos operando sean 1
\end{ejem}
\end{frame}

\begin{frame}
\frametitle{Números naturales}
\begin{ejem}
Sea la operación binaria OR ( $||$ ) que establece una operación lógica bajo el conjunto $S = \{0,1\}$, tendremos la siguiente tabla de resultados (recordemos que es verdadero o 1 cuando al menos uno de los operandos es verdadero, en este caso 1).
\begin{center}
\begin{tabular}{| r | c | l |}
\hline
\textbf{$||$} & \textbf{0} & \textbf{1}	\\ \hline
\textbf{0} & $0$ & $1$\\
\textbf{1} & $1$ & $1$\\ \hline
\end{tabular}
\end{center}
Es fácil ver que existe la propiedad conmutativa dada esta operación bajo este conjunto pues $1 ~ || ~ 0 = 0 ~ || ~ 1 = 1$\\
También se posee la propiedad asociativa ya que no importa el orden en que agrupemos y operemos, pues sólo será 0 cuando todos los elementos que estamos operando sean 0
\end{ejem}
\end{frame}

\begin{frame}
\frametitle{Números naturales}
\begin{ejem}
Tomaremos las dos operaciones anteriores AND y OR para determinar que AND es distributiva sobre OR.\\
\begin{center}
$a~\&~(b~||~c) = (a~\&~b)~||~(a~\&~c)~\forall~a,b,c~\in~\{0,1\}$
\end{center}
Si lo pensamos bien, es fácil deducirlo, ya que $(b~||~c)$ se convierte en un operando $d$ que será verdadero si al menos uno es verdadero, así que si los dos son falsos $(a~\&~b) = (a~\&~c) = d = 0$ sin importar el valor de $a$, pero en el caso de que alguno sea verdadero entonces $(a~\&~b)$ y/ó $(a~\&~c)$ serán 1, siendo así que el resultado será 1
\end{ejem}
\end{frame}

\begin{frame}
\frametitle{Números enteros}
\framesubtitle{Propiedades de la suma y el producto}
\begin{defi}
Se conoce como números enteros al conjunto numérico que contiene al conjunto $\mathbb{N}$, a sus inversos negativos
y al cero. Se denota de la siguiente forma:\\
\begin{center}
$\mathbb{Z} = \{...,-3,-2,-1,0,1,2,3,...\}$
\end{center} 
\end{defi}
\begin{eje}
Demostrar que las propiedades asociativa, conmutativa y distributiva se cumplen para la multiplicación y adición bajo el 
conjunto $\mathbb{Z}$ 
\end{eje}
\end{frame}

\begin{frame}
\frametitle{Números enteros}
\framesubtitle{Elemento neutro}
\begin{defi}
Sea i un número que pertenece a un conjunto de números $S$ y $\oplus$ una operación binaria bajo el conjunto $S$, 
se dice que i es el elemento neutro o elemento identidad si se cumple:\\
\begin{center}
$ a \oplus i = i \oplus a = a~\forall ~a~\in S$
\end{center}
\end{defi}
\begin{itemize}
\item El elemento neutro para la adición bajo $\mathbb{Z}$ es el número 0\\
$a + 0 = 0 + a = a~\forall~a~\in~\mathbb{Z}$
\item El elemento neutro para el producto bajo $\mathbb{Z}$ es el número 1
$a \cdot 1 = 1 \cdot a = a~\forall~a~\in~\mathbb{Z}$
\end{itemize}
\end{frame}

\begin{frame}
\frametitle{Números enteros}
\framesubtitle{Inverso aditivo}
\begin{defi}
Sean $a,-a~\in~\mathbb{Z}$, se dice que uno es inverso aditivo del otro si se cumple que $a + (-a) = (-a) + a = 0$
\end{defi} 
\end{frame}

\begin{frame}
\frametitle{Números enteros}
\begin{ejem}
Sea el conjunto $S = \{a,b,c,d,e\}$ y la operación $\oplus$ cuya regla de correspondencia está dada por la siguiente tabla: \\
\begin{center}
\begin{tabular}{| r | c | c | c | c | l |}
\hline
\textbf{$\oplus$} & \textbf{a} & \textbf{b} & \textbf{c} & \textbf{d} & \textbf{e}	\\ \hline
\textbf{a} & a & b & c & d & e \\
\textbf{b} & b & e & a & c & d \\
\textbf{c} & c & a & b & e & d \\ 
\textbf{d} & d & c & e & c & a \\ 
\textbf{e} & e & d & d & a & d \\ \hline
\end{tabular}
\end{center}
Podemos observar que esta operación binaria posee elemento neutro ya que al operar con \textbf{a} obtenemos como resultado el valor del otro operando.\\
También existen los inversos para todos los elementos de $S$, ya que $b~\oplus~c = c~\oplus~b = d~\oplus~e = e~\oplus~d = a~\oplus~a = a$, donde cada par de operandos son inversos entre sí.
\end{ejem}
\end{frame}

\begin{frame}
\begin{eje}
Del ejemplo anterior, determinar si posee las propiedades conmutativa y asociativa
\end{eje}
\end{frame}

\begin{frame}
\frametitle{Números racionales}
\framesubtitle{Propiedades de la suma y el producto}
\begin{defi}
Los números racionales son el conjunto de números que puede expresarse como el cociente de dos números enteros y se
representa de la siguiente forma:\\
\begin{center}
$\mathbb{Q} = \{r = (p/q): p,q\in~\mathbb{Z}\}$
\end{center}
\end{defi} 
\begin{eje}
Demostrar que las propiedades asociativa, conmutativa y distributiva se cumplen para la multiplicación y adición bajo el 
conjunto $\mathbb{Q}$ 
\end{eje}
\end{frame}

\begin{frame}
\frametitle{Números racionales}
\framesubtitle{Inversos multiplicativos}
\begin{defi}
El inverso multiplicativo o recíproco de un número $x$ se expresa como $\frac{1}{x}$ ó $x^{-1}$ y se cumple que 
$\frac{1}{x} \cdot x = x^{-1} \cdot x = 1$
\end{defi}
\end{frame}

\begin{frame}
\frametitle{Números reales}
\framesubtitle{Propiedades de la suma y el producto}
\begin{defi}
Los números reales son el conjunto resultado de la unión de los números racionales e irracionales (aquellos que no 
pueden ser expresados como el cociente de dos enteros) y se denota de la siguiente forma:
\begin{center}
$\mathbb{R} = \mathbb{Q} \cup \mathbb{Q}^{c}$
\end{center}
\end{defi}
\begin{eje}
Demostrar que las propiedades asociativa, conmutativa y distributiva se cumplen para la multiplicación y adición bajo el 
conjunto $\mathbb{R}$ 
\end{eje}
\end{frame}

\begin{frame}
\frametitle{Números reales}
\framesubtitle{Completitud}
\begin{axiom}
Cada conjunto no vacío de números reales que está acotado superiormente tiene una cota superior mínima.
\\Esto es $\forall ~conjunto ~ S \neq ~ \emptyset, S~\subseteq~\mathbb{R}~\exists~b~\in~\mathbb{R}:~b\geqslant a~\forall~
a~\in~S$
\end{axiom}
Se puede plantear el mismo caso pero para definir un conjunto acotado inferiormente por una cota superior máxima.
\\A esta cota superior mínima se le conoce como \textbf{supremo} y es único, por lo que si se propone un supremo $S_1~y~S_2$
siempre se llegará a la conclusión $S_1 = S_2$
\end{frame}

\begin{frame}
\frametitle{Números reales}
\framesubtitle{Estructuras algebraicas}
\begin{defi}
Una \textbf{estructura algebraica} es un conjunto de números $S$ no vacío asociado a sus respectivas operaciones 
aplicables sobre los elementos que lo componen. 
\end{defi}
Ejemplo:
\\$(\mathbb{R},+,*)$ \hspace{2cm}
\\ 
De manera general podemos dividir a las estructuras algebraicas en tres tipos:
\begin{itemize}
\item Grupos
\item Anillos
\item Campos
\end{itemize}
\end{frame}

\begin{frame}
\frametitle{Números reales}
\framesubtitle{Grupos}
\begin{defi}
Sea G un conjunto de números que posee la propiedad de cerradura para una operación $\oplus$, se dice que es un grupo
si se cumple que:
\begin{itemize}
\item Posee la propiedad asociativa
\item Tiene elemento neutro
\item Existe elemento inverso para todos los elementos de G
\end{itemize}
\end{defi}
\end{frame}

\begin{frame}
\frametitle{Números reales}
\begin{ejem}
Un ejemplo de \textbf{grupo} es la tupla $(\mathbb{Z},+)$, pues la suma es cerrada para los números enteros, es decir, la suma de n enteros siempre dará como resultado un entero. Anteriormente se propuso como ejercicio demostrar que posee la propiedad asociativa, así que daremos por hecho que ya está demostrado. \\
Existe un elemento neutro que es el 0, pues $a + 0 = 0 + a = a~\forall~a\in~\mathbb{Z}$\\
También existe un inverso para cada elemento de $\mathbb{Z}$, pues es el mismo número pero con signo contrario, es decir, $\forall~a, \in~\mathbb{Z}~\exists -a: a + (-a) = (-a) + a = 0$\\
Así podemos concluir que es un grupo, pero además es un \textbf{grupo conmutativo o abeliano} pues también posee esa propiedad.
\end{ejem}
\end{frame}


\begin{frame}
\frametitle{Números reales}
\framesubtitle{Anillo}
\begin{defi}
Un anillo es una estructura algebraica compuesta por un conjunto G donde se definen las operaciones '$+,\cdot$' y se cumple lo siguiente: 
\\
\begin{itemize}
\item La adición es asociativa
\item La adición es conmutativa
\item Existe el neutro para la adición
\item Existe inverso aditivo para todos los elementos de G
\item El producto es asociativo
\item El producto es distributivo sobre la suma
\end{itemize}
\end{defi}
\end{frame}

\begin{frame}
\frametitle{Números reales}
\begin{ejem}
Siguiendo el ejemplo anterior de grupo, incluiremos a la tupla la operación '$\cdot$', teniendo así $(\mathbb{Z},+,\cdot)$\\
Cabe destacar que las condiciones respecto a la suma se resumen a que $(\mathbb{Z},+)$ sea un grupo abeliano o conmutativo, algo que ya determinamos anteriormente. Por lo que falta verficar las propiedades sobre el producto.\\
Cuando vimos los números enteros, se dejó como ejercicio demostrar la asociatividad y distributividad del producto sobre la suma, así que podemos concluir que es un \textbf{anillo conmutativo}, ya que el producto también posee esa propiedad. Más adelante veremos que esta tupla es una estructura más compleja cumpliendo otras propiedades que lo convierten en un campo.
\end{ejem}
\end{frame}

\begin{frame}
\frametitle{Números reales}
\framesubtitle{Campo}
\begin{defi}
Un cuerpo o campo es una estructura algebraica compuesta por un conjunto G donde se definen las operaciones '$+,\cdot$' y se cumple lo siguiente: 
\\
\begin{itemize}
\item La adición es asociativa
\item La adición es conmutativa
\item Existe el neutro para la adición
\item Existe inverso aditivo para todos los elementos de G
\item El producto es asociativo
\item El producto es conmutativo
\item Existe el neutro para el producto
\item Existe inverso multiplicativo para todos los elementos de G
\item El producto es distributivo sobre la suma
\end{itemize}
\end{defi}
\end{frame}

\begin{frame}
\frametitle{Números reales}
\begin{ejem}
Como ya vimos, las caracterísitcas de la suma se resumen a que $(\mathbb{Z},+)$ sea un grupo abeliano o conmutativo y en este caso debemos verificar lo mismo para el producto, es decir, $(\mathbb{Z},\cdot)$ sea un grupo abeliano o conmutativo, sin embargo, a lo largo de la unidad hemos visto que se cumplen todas esas propiedades para el producto (conmutatividad, asocitividad, neutro = 1, inverso = $\frac{1}{x}$, distributividad), por lo que concluimos que además de ser un anillo abeliano, cumple los requisitos para ser un \textbf{campo}. 
\end{ejem}
\begin{eje}
Demostrar que $(\mathbb{R},+,\cdot)$ es un campo
\end{eje}
\end{frame}

\end{document}