\documentclass[11pt]{beamer}
\usetheme{Madrid}
\usepackage[utf8]{inputenc}
\usepackage[spanish]{babel}
\usepackage{amsmath}
\usepackage{amsfonts}
\usepackage{amssymb}
\usepackage{graphicx}
\usepackage{amsthm}
\newtheorem{defi}{Definición}
\newtheorem{eje}{Ejercicio}
\newtheorem{ejem}{Ejemplo}
\newtheorem{axiom}{Axioma}
\newtheorem{teor}{Teorema}
\usepackage{lipsum}
\author{Adriana Dávila Santos}
\title{Operaciones con Matrices}
%\setbeamercovered{transparent} 
%\setbeamertemplate{navigation symbols}{} 
%\logo{} 
%\institute{} 
%\date{} 
%\subject{} 
\begin{document}

\begin{frame}
\titlepage
\end{frame}

%\begin{frame}
%\tableofcontents
%\end{frame}

\begin{frame}{• Objetivo particular:}
El alumno realizará sumas y productos de matrices e identificará las propiedades de estas
operaciones, determinará la transpuesta de una matriz y obtendrá, en caso de que exista, la
inversa de una matriz por medio de operaciones elementales.
\end{frame}

\begin{frame}
\frametitle{Operaciones fundamentales con matrices}
\framesubtitle{Suma y producto por un escalar}
\begin{defi}
Sea $M$ el conjunto de matrices de orden $mxn:~m,n\in ~ \mathbb{N}$, definimos a la operación binaria \textbf{suma} $(+)$ bajo este conjunto para dos matrices cualesquiera A y B si se cumple que son del mismo orden. El resultado será una mtriz C del mismo orden que A y B dónde los elementos se calculan de la siguiente forma: \\
\begin{center}
$c_{ij} = a_{ij} + b_{ij}$
\end{center}
\end{defi}
\begin{ejem}
${\displaystyle \mathbf {A} ={\begin{pmatrix}2&3\\5&6\\\end{pmatrix}}}$, 
${\displaystyle \mathbf {B} ={\begin{pmatrix}4&1\\8&0\\\end{pmatrix}}}$, 
${\displaystyle \mathbf {A+B=C} ={\begin{pmatrix}6&4\\13&6\\\end{pmatrix}}}$
\end{ejem}
\end{frame}

\begin{frame}
\frametitle{Operaciones fundamentales con matrices}
\framesubtitle{Suma y producto por un escalar}
\begin{defi}
Sea $k$ una constante y la matriz A de orden $mxn$, definimos el \textbf{producto por un escalar} a la multiplicación de la constante $k$ por la matriz A, dando como resultado una matriz $kA = B$ de orden mxn donde sus elementos se calculan de la siguiente forma:\\
\begin{center}
$b_{ij} = k\cdot a_{ij}$
\end{center}
\end{defi}
\begin{ejem}
$k = 2, ~ {\displaystyle \mathbf {A} ={\begin{pmatrix}2&3\\5&6\\\end{pmatrix}}}$,
${\displaystyle \mathbf {2A = B} ={\begin{pmatrix}4&6\\10&12\\\end{pmatrix}}}$ 
\end{ejem}
\end{frame}

\begin{frame}
\frametitle{Operaciones fundamentales con matrices}
\framesubtitle{Propiedades de la suma y producto por un escalar}
\textbf{Propiedades de la suma:}
\begin{itemize}
\item Conmutatividad
\item Asociatividad
\item Según el conjunto bajo el que se trabaje, existe el neutro único
\item Según el conjunto bajo el que se trabaje, existe el inverso\\ \hspace{0cm} \\
\end{itemize}
\textbf{Propiedades del producto por un escalar:}\\
En este caso trabajamos con elementos de dos conjuntos diferentes por lo que no definimos propiedades como en una operación binaria donde todos los elementos pertenecen al mismo conjunto.
\end{frame}

\begin{frame}
\frametitle{Operaciones fundamentales con matrices}
\framesubtitle{Transpuesta de una matriz}
\begin{defi}
Sea A una matriz de orden $mxn$, decimos que su \textbf{transpuesta} será la matriz $A^T$ de orden $nxm$ cuyos elementos se calculan de la siguiente forma:\\
\begin{center}
$a^{T}_{ij} = a_{ji}$
\end{center}
\end{defi}
\begin{ejem}
${\displaystyle \mathbf {A} ={\begin{pmatrix}4&5&1\\8&7&9\\\end{pmatrix}}}$, 
${\displaystyle \mathbf {A^T} ={\begin{pmatrix}4&8\\5&7\\1&9\\\end{pmatrix}}}$ 
\end{ejem}
\end{frame}

\begin{frame}
\frametitle{Operaciones fundamentales con matrices}
\framesubtitle{Transpuesta de una matriz}
\begin{defi}
Sea A una matriz de orden n, se dice que A es \textbf{simétrica} si y sólo si $A=A^T$. Es fácil ver que para ser simétrica debe ser cuadrada.
\end{defi}
\begin{ejem}
${\displaystyle \mathbf {A} ={\begin{pmatrix}0&1&-1\\1&0&2\\-1&2&1\\\end{pmatrix}}}$ = 
${\displaystyle \mathbf {A^T} ={\begin{pmatrix}0&1&-1\\1&0&2\\-1&2&1\\\end{pmatrix}}}$ 
\end{ejem}
\end{frame}

\begin{frame}
\frametitle{Operaciones fundamentales con matrices}
\framesubtitle{Transpuesta de una matriz}
\textbf{Propiedades de la transpuesta: }\\
\begin{itemize}
\item Si la matriz A es cuadrada y diagonal, $A = A^T$
\item La transpuesta de la transpuesta de A es A: $(A^T)^T = A$
\item La transpuesta de la suma de matrices es $(A + B)^T = A^T + B^T$
\item La transpuesta del producto de un escalar $\alpha$ por una matriz A es $(\alpha \cdot A)^T = \alpha \cdot A^T$
\item La transpuesta del producto de matrices es $(A·B)^T = B^T \cdot A^T$
\end{itemize}
\begin{eje}
Demuestre las propiedades anteriores a excepción de la última
\end{eje}
\end{frame}

\begin{frame}
\frametitle{Operaciones fundamentales con matrices}
\framesubtitle{Transpuesta de una matriz}
\begin{defi}
La matriz \textbf{transpuesta conjugada} de una matriz A de orden $mxn$ es la matriz resultante de obtener todos los conjugados para $a_{ij}$ y después transponer esa matriz o viceversa. Esto solo tiene sentido si los coeficientes $a_{ij}$ son números complejos donde su parte imaginaria es diferente de 0.
\end{defi}
\begin{ejem}
Una matriz ${\displaystyle A={\begin{pmatrix}2i&6-i\\3+i&4\end{pmatrix}}}$ tiene el transpuesto conjugado ${\displaystyle A^{*}={\begin{pmatrix}-2i&3-i\\6+i&4\end{pmatrix}}}$
\end{ejem}
\end{frame}

\begin{frame}
\frametitle{Operaciones fundamentales con matrices}
\framesubtitle{Transpuesta de una matriz}
\begin{defi}
Una matriz \textbf{hermitiana} es una matriz cuadrada A de elementos complejos que tiene la característica de ser igual a su transpuesta conjugada $A^*$.
\end{defi}
\begin{ejem}
${\displaystyle A={\begin{bmatrix}3&2+i\\2-i&1\end{bmatrix}}}$ = 
${\displaystyle A^*={\begin{bmatrix}3&2+i\\2-i&1\end{bmatrix}}}$
\end{ejem}
\end{frame}

\begin{frame}
\frametitle{Operaciones fundamentales con matrices}
\framesubtitle{Producto de matrices}
\begin{defi}
Sean las matrices A de orden $mxn$ y B de orden $nxp$, el producto AB será una matriz C de orden $mxp$ cuyos elementos se calculan de la siguiente forma:\\
\begin{center}
$c_{ij} =
\sum_{k=1}^{n}a_{ik}b_{kj} $
\end{center}
\end{defi}
\begin{ejem}
${\displaystyle A={\begin{pmatrix}2&6\\3&4\end{pmatrix}}}$, 
${\displaystyle B={\begin{pmatrix}3&5\\2&3\end{pmatrix}}}$, \\ \hspace{0cm} \\
${\displaystyle AB=C={\begin{pmatrix}18&28\\17&27\end{pmatrix}}}$
\end{ejem}
\end{frame}

\begin{frame}
\frametitle{Operaciones fundamentales con matrices}
\framesubtitle{Inversa de una matriz}
\begin{defi}
Sea la matriz A de orden n, se dice que es una matriz \textbf{invertible} o \textbf{regular} si existe otra matriz cuadrada del mismo orden llamada \textbf{inversa} de A y denotada como $A^{-1}$ de modo que se cumpla que $AA^{-1} = I_n$ (matriz identidad de orden n)
\end{defi}
\begin{ejem}
La primer matriz será A, por lo tanto la segunda será $A^{-1}$, ya que se cumple que su producto es igual a la matriz identidad de orden 2.\\ \hspace{0cm} \\
${\displaystyle {\begin{bmatrix}2&1\\5&3\end{bmatrix}}{\begin{bmatrix}3&-1\\-5&2\end{bmatrix}}={\begin{bmatrix}1&0\\0&1\end{bmatrix}}}$
\end{ejem}
\end{frame}

\begin{frame}
\frametitle{Operaciones fundamentales con matrices}
\framesubtitle{Inversa de una matriz}
Este tipo de matrices se usa para resolver ecuaciones con matrices o para resolver sistemas de ecuaciones lineales, de ahí su importancia.\\ \hspace{0cm} \\
Para que una matriz A tenga inversa debe ser cuadrada de orden n, $n\in \mathbb{N}$ y el rango de A, r(A) = 0.
\end{frame}

\begin{frame}
\frametitle{Operaciones fundamentales con matrices}
\framesubtitle{Inversa de una matriz}
Vamos a explicar el proceso para obtener la matriz inversa por medio de operaciones elementales por rengón, para ello usaremos un ejemplo.\\ \hspace{0cm} \\
Obtenga la inversa de la siguiente matriz:\\
\begin{center}
${\displaystyle A = {\begin{bmatrix}1&0&2\\2&-1&3\\4&1&8\end{bmatrix}}}$\\ \hspace{0cm} \\
\end{center}
Lo primero que vamos a hacer es aumentar la matriz del lado derecho, colocando una equivalente a $I_n$, es decir, una matriz identidad de orden 3 en este caso.
\end{frame}

\begin{frame}
\frametitle{Operaciones fundamentales con matrices}
\framesubtitle{Inversa de una matriz}
\begin{center}
${\displaystyle {\begin{bmatrix}1&0&2&|&1&0&0\\2&-1&3&|&0&1&0\\4&1&8&|&0&0&1\end{bmatrix}}}$\\ \hspace{0cm} \\
\end{center}
Ahora lo que sigue es aplciar operaciones elementales por rengón de modo que del lado izquierdo tengamos la matriz $I_3$ y la matriz que quede del lado derecho será la inversa de A.\\ \hspace{0cm} \\
${\displaystyle {\begin{bmatrix}1&0&2&|&1&0&0\\2&-1&3&|&0&1&0\\4&1&8&|&0&0&1\end{bmatrix}}} 
~ R_2 \rightarrow R_2 -2R_1 ~ {\displaystyle {\begin{bmatrix}1&0&2&|&1&0&0\\0&-1&-1&|&-2&1&0\\4&1&8&|&0&0&1\end{bmatrix}}}$\\ \hspace{0cm} \\
\end{frame}

\begin{frame}
\frametitle{Operaciones fundamentales con matrices}
\framesubtitle{Inversa de una matriz}
${\displaystyle {\begin{bmatrix}1&0&2&|&1&0&0\\0&-1&-1&|&-2&1&0\\4&1&8&|&0&0&1\end{bmatrix}}} 
~ R_3 -4R_1 \rightarrow ~ {\displaystyle {\begin{bmatrix}1&0&2&|&1&0&0\\0&-1&-1&|&-2&1&0\\0&1&0&|&-4&0&1\end{bmatrix}}}$\\ \hspace{0cm} \\
${\displaystyle {\begin{bmatrix}1&0&2&|&1&0&0\\0&-1&-1&|&-2&1&0\\0&1&0&|&-4&0&1\end{bmatrix}}} 
~ R_3 +R_2 \rightarrow ~ {\displaystyle {\begin{bmatrix}1&0&2&|&1&0&0\\0&-1&-1&|&-2&1&0\\0&0&-1&|&-6&1&1\end{bmatrix}}}$\\ \hspace{0cm} \\
${\displaystyle {\begin{bmatrix}1&0&2&|&1&0&0\\0&-1&-1&|&-2&1&0\\0&0&-1&|&-6&1&1\end{bmatrix}}} 
~ -R_2, -R_3 \rightarrow ~ {\displaystyle {\begin{bmatrix}1&0&2&|&1&0&0\\0&1&1&|&2&-1&0\\0&0&1&|&6&-1&-1\end{bmatrix}}}$\\ \hspace{0cm} \\
${\displaystyle {\begin{bmatrix}1&0&2&|&1&0&0\\0&1&1&|&2&-1&0\\0&0&1&|&6&-1&-1\end{bmatrix}}} 
~ R_2-R_3 \rightarrow ~ {\displaystyle {\begin{bmatrix}1&0&2&|&1&0&0\\0&1&0&|&-4&0&1\\0&0&1&|&6&-1&-1\end{bmatrix}}}$\\ \hspace{0cm} \\
\end{frame}

\begin{frame}
\frametitle{Operaciones fundamentales con matrices}
\framesubtitle{Inversa de una matriz}
${\displaystyle {\begin{bmatrix}1&0&2&|&1&0&0\\0&1&0&|&-4&0&1\\0&0&1&|&6&-1&-1\end{bmatrix}}} 
~ R_1-2R_3 \rightarrow ~ {\displaystyle {\begin{bmatrix}1&0&0&|&-11&2&2\\0&1&0&|&-4&0&1\\0&0&1&|&6&-1&-1\end{bmatrix}}}$\\ \hspace{0cm} \\
Como podemos observar, del lado izquiero tenemos la matriz $I_3$ por lo que detenemos el proceso y conlcuimos que:\\ \hspace{0cm} \\
\begin{center}
${\displaystyle A^{-1}={\begin{bmatrix}-11&2&2\\-4&0&1\\6&-1&-1\end{bmatrix}}}$\\ \hspace{0cm} \\
\end{center}
Comprobamos de la siguiente forma: \\ \hspace{0cm} \\
${\displaystyle {\begin{bmatrix}1&0&2\\2&-1&3\\4&1&8\end{bmatrix}}}$
${\displaystyle {\begin{bmatrix}-11&2&2\\-4&0&1\\6&-1&-1\end{bmatrix}}}$ = 
${\displaystyle {\begin{bmatrix}1&0&0\\0&1&0\\0&0&1\end{bmatrix}}}$\\ \hspace{0cm} \\
\end{frame}

\end{document}