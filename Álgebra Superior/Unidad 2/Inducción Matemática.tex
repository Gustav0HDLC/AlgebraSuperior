\documentclass[11pt]{beamer}
\usetheme{Madrid}
\usepackage[utf8]{inputenc}
\usepackage[spanish]{babel}
\usepackage{amsmath}
\usepackage{amsfonts}
\usepackage{amssymb}
\usepackage{graphicx}
\usepackage{amsthm}
\newtheorem{defi}{Definición}
\newtheorem{eje}{Ejercicio}
\newtheorem{ejem}{Ejemplo}
\newtheorem{axiom}{Axioma}
\usepackage{Lipsum}
\author{Adriana Dávila Santos}
\title{Inducción Matemática}
%\setbeamercovered{transparent} 
%\setbeamertemplate{navigation symbols}{} 
%\logo{} 
%\institute{} 
%\date{} 
%\subject{} 
\begin{document}

\begin{frame}
\titlepage
\end{frame}
%\begin{frame}
%\tableofcontents
%\end{frame}

\begin{frame}{• Objetivo particular}
El alumno demostrará proposiciones acerca de los números naturales por medio de inducción
matemática.
\end{frame}

\begin{frame}
\frametitle{Los Postulados de Peano: el principio de inducción}
\begin{itemize}
\item $1~\in~\mathbb{N}$
\item $\forall~n~\in~\mathbb{N},~\exists$ un único $n'$ : denominado \textbf{sucesor de n} ($n + 1 = n'$)
\item $\forall~n~\in~\mathbb{N}$ se tiene que $n'~\neq~1$ (1 no es sucesor en los naturales)
\item $Si~m,n~\in~\mathbb{N}~\&~m' = n'~\rightarrow~m = n$
\item Sea $S~\subseteq~\mathbb{N} : 1~\in~S~\&~\forall~k~\in~\mathbb{S}~\exists~k'~\rightarrow~S = \mathbb{N}$
\end{itemize} 
Al último postulado se le conoce como \textbf{Principio de inducción}
\end{frame}

\begin{frame}
\frametitle{Inducción Matemática}
La inducción matemática es un método de demostración muy útil para probar propiedades, principalmente en los números enteros,
así como probar algunas simplificaciones para series de números.
De manera general este método consta de 3 pasos:
\begin{itemize}
\item \textbf{Caso base}\\
Debemos partir de un valor para el que se cumpla la propiedad o serie que queremos demostrar, por lo general empezamos 
con $n = 1$
\item \textbf{n = k}\\
Ya que probamos que se cumple para un caso base, asumimos que se cumple para cualquier $k~\in~\mathbb{N}$,
\item \textbf{n = k+1}\\
Sólo queda por demostrar que la propiedad se cumple para $n = k+1$
\end{itemize}
\end{frame}

\begin{frame}
\frametitle{Inducción Matemática}
\begin{ejem}
Determine si la siguiente igualdad es verdadera por el método de inducción:\\ \hspace{0cm} \\
$1 + 2 + 3 + ... + n = \frac{n(n + 1)}{2}$\\ \hspace{0cm} \\
\begin{itemize}
\item Paso 1\\
Demostramos que se cumpla para $n = 1$\\ \hspace{0cm} \\
$1 = \frac{1(1 + 1)}{2} = \frac{2}{2} = 1$\\ \hspace{0cm} \\
\item Paso 2\\
Se cumple para el caso base, así que asumimos que se cumple para $n = k$\\ \hspace{0cm} \\
$1 + 2 + 3 + ... + k = \frac{k(k + 1)}{2}$
\end{itemize}
\end{ejem}
\end{frame}

\begin{frame}
\frametitle{Inducción Matemática}
\begin{ejem}
\begin{itemize}
\item Paso 3\\
Sólo queda por demostrar que se cumpla para $n = k+1$, es decir, verificar la siguiente igualdad\\ \hspace{0cm} \\
$\frac{k(k + 1)}{2} + (k+1) = \frac{(k+1)[(k+1)+1]}{2}$\\ \hspace{0cm} \\
$\frac{k(k + 1)+2(k+1)}{2} = \frac{(k+1)(k+2)}{2}$\\ \hspace{0cm} \\
$\frac{k^2+3k+2}{2} = \frac{k^2+3k+2}{2} ~ ~ _Q ~ _E ~ _D$\\ \hspace{0cm} \\
NOTA: Lo más formal es sólo operar el lado derecho y pasarlo a la forma del lado izquierdo, sin embargo en estos ejemplos operaremos de ambos lados de la igualdad, simplificando y usando las propiedades que nos convengan pues es igual de válida la demostración.
\end{itemize}
\end{ejem}
\end{frame}

\begin{frame}
\frametitle{Inducción Matemática}
\begin{ejem}
Determine si la siguiente igualdad es verdadera por el método de inducción:\\ \hspace{0cm} \\
$8 + 11 + 14 + ... + (3n + 5) = \frac{n(3n + 13)}{2}$\\ \hspace{0cm} \\
\begin{itemize}
\item Paso 1\\
Demostramos que se cumpla para $n = 1$\\ \hspace{0cm} \\
$(3\cdot1 + 5) = \frac{1(3\cdot1 + 13)}{2} = \frac{16}{2} = 8$\\ \hspace{0cm} \\
\item Paso 2\\
Se cumple para el caso base, así que asumimos que se cumple para $n = k$\\ \hspace{0cm} \\
$8 + 11 + 14 + ... + (3k + 5) = \frac{k(3k + 13)}{2}$\\ \hspace{0cm} \\
\end{itemize}
\end{ejem}
\end{frame}

\begin{frame}
\frametitle{Inducción Matemática}
\begin{ejem}
\begin{itemize}
\item Paso 3\\
Sólo queda por demostrar que se cumpla para $n = k+1$, es decir, verificar la siguiente igualdad\\ \hspace{0cm} \\
$\frac{k(3k + 13)}{2} + [3(k+1)+5] = \frac{(k+1)[3(k+1) + 13]}{2}$\\ \hspace{0cm} \\
$\frac{k(3k + 13)+2(3k+8)}{2} = \frac{(k+1)(3k+16)}{2}$\\ \hspace{0cm} \\
$\frac{3k^2+19k+16}{2} = \frac{3k^2+19k+16}{2} ~ ~ _Q ~ _E ~ _D$\\ \hspace{0cm} \\
\end{itemize}
\end{ejem}
\end{frame}


\begin{frame}
\frametitle{Inducción Matemática}
\begin{ejem}
Determine si la siguiente igualdad es verdadera por el método de inducción:\\ \hspace{0cm} \\
$1^2 + 2^2 + 3^2 + ... + n^2 = \frac{n(n + 1)(2n + 1)}{6}$\\ \hspace{0cm} \\
\begin{itemize}
\item Paso 1\\
Demostramos que se cumpla para $n = 1$\\ \hspace{0cm} \\
$1^2 = \frac{1(1 + 1)(2\cdot1 + 1)}{6} = \frac{6}{6} = 1$
\item Paso 2\\
Se cumple para el caso base, así que asumimos que se cumple para $n = k$\\ \hspace{0cm} \\
$1^2 + 2^2 + 3^2 + ... + k^2 = \frac{k(k + 1)(2k + 1)}{6}$\\ \hspace{0cm} \\
\end{itemize}
\end{ejem}
\end{frame}

\begin{frame}
\frametitle{Inducción Matemática}
\begin{ejem}
\begin{itemize}
\item Paso 3\\
Sólo queda por demostrar que se cumpla para $n = k+1$, es decir, verificar la siguiente igualdad\\ \hspace{0cm} \\
$\frac{k(k + 1)(2k + 1)}{6} + (k+1)^2 = \frac{(k+1)[(k+1) + 1][2(k + 1)+1]}{6}$\\ \hspace{0cm} \\
$\frac{(k^2 + k)(2k + 1)+6(k+1)^2}{6} = \frac{(k+1)(k+2)(2k + 3)}{6}$\\ \hspace{0cm} \\
$\frac{2k^3+3k^2+k+6(k^2+2k+1)}{6} = \frac{(k+1)(2k^2+7k+6)}{6}$\\ \hspace{0cm} \\
$\frac{2k^3+9k^2+13k+6}{6} = \frac{2k^3+9k^2+13k+6}{6} ~ ~ _Q ~ _E ~ _D$\\ \hspace{0cm} \\
\end{itemize}
\end{ejem}
\end{frame}

\begin{frame}
\begin{eje}
Determine si la siguiente igualdad es verdadera por el método de inducción:\\ \hspace{0cm} \\
$1^2+3^2+5^2+...+(2n-1)^2 = \frac{n(2n-1)(2n+1)}{3}$
\end{eje}
\end{frame}

\end{document}